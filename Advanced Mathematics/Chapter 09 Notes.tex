%!TEX program = xelatex
\documentclass[12pt, a4paper]{article}

\usepackage[dvipsnames]{xcolor}

\usepackage{fancyhdr}
\usepackage{extramarks}
\usepackage{amsmath}
\usepackage{empheq}
\usepackage{amsthm}
\usepackage{amsfonts}
\usepackage{tikz}
\usepackage{tikz-3dplot}
\usepackage[plain]{algorithm}
\usepackage{algpseudocode}

\usepackage{ctex}
\usepackage{upgreek}
\usepackage{indentfirst}
\usepackage{wrapfig}
\usepackage{subfigure}
\usepackage{pgfplots}
\usepgfplotslibrary{patchplots}
\usepgfplotslibrary{colormaps}
\usepgfplotslibrary{colorbrewer}
\pgfplotsset{compat=1.18}
\usetikzlibrary{automata,positioning,shapes.geometric,arrows.meta,patterns,calc}
\numberwithin{equation}{section}
\CTEXoptions[today=old]

%
% Basic Document Settings
%

\topmargin=-0.25in
\evensidemargin=0in
\oddsidemargin=0in
\textwidth=6.5in
\textheight=9.2in
\headsep=0.25in

\linespread{1.1}

\pagestyle{fancy}
\lhead{\hmwkAuthorName}
\chead{\hmwkClass : \hmwkTitle}
\rhead{\firstxmark}
\lfoot{\lastxmark}
\cfoot{\thepage}

\renewcommand\headrulewidth{0.4pt}
\renewcommand\footrulewidth{0.4pt}

\setlength{\parindent}{2em}  % 2em代表首行缩进两个字符

%
% Create Problem Sections
%

\newcommand{\enterProblemHeader}[1]{
    \nobreak\extramarks{}{Problem \arabic{#1} continued on next page\ldots}\nobreak{}
    \nobreak\extramarks{Problem \arabic{#1} (continued)}{Problem \arabic{#1} continued on next page\ldots}\nobreak{}
}

\newcommand{\exitProblemHeader}[1]{
    \nobreak\extramarks{Problem \arabic{#1} (continued)}{Problem \arabic{#1} continued on next page\ldots}\nobreak{}
    \stepcounter{#1}
    \nobreak\extramarks{Problem \arabic{#1}}{}\nobreak{}
}

% \setcounter{secnumdepth}{0}
\newcounter{partCounter}
\newcounter{homeworkProblemCounter}
\setcounter{homeworkProblemCounter}{0}
% \nobreak\extramarks{Problem \arabic{homeworkProblemCounter}}{}\nobreak{}

%
% Homework Problem Environment
%
% This environment takes an optional argument. When given, it will adjust the
% problem counter. This is useful for when the problems given for your
% assignment aren't sequential. See the last 3 problems of this template for an
% example.
%
\newenvironment{homeworkProblem}[1][-1]{
    \ifnum#1>0
        \setcounter{homeworkProblemCounter}{#1}
    \fi
    \section{Problem \arabic{homeworkProblemCounter}}
    \setcounter{partCounter}{1}
    \enterProblemHeader{homeworkProblemCounter}
}{
    \exitProblemHeader{homeworkProblemCounter}
}

%
% Homework Details
%   - Title
%   - Due date
%   - Class
%   - Section/Time
%   - Instructor
%   - Author
%

\newcommand{\hmwkTitle}{The Method of Differentiation of Multivariate Functions and Its Applications}
\newcommand{\hmwkClass}{Advanced Mathematics}
\newcommand{\hmwkClassTime}{}
\newcommand{\myUniversiy}{Wuhan University}
\newcommand{\hmwkAuthorName}{\textbf{Lai Wei}}

%
% Title Page
%

\title{
    \vspace{2in}
    \textmd{\textbf{\hmwkClass:\ \hmwkTitle}}\\
    \vspace{0.4in}
    \large{\textit{\myUniversiy}}
    \vspace{3in}
}

\author{\hmwkAuthorName}
\date{\today}

\renewcommand{\part}[1]{\textbf{\large Part \Alph{partCounter}}\stepcounter{partCounter}\\}

%
% Various Helper Commands
%

% Useful for algorithms
\newcommand{\alg}[1]{\textsc{\bfseries \footnotesize #1}}

% % For derivatives
% \newcommand{\deriv}[1]{\frac{\mathrm{d}}{\mathrm{d}x} (#1)}

% For partial derivatives
\newcommand{\pderiv}[2]{\frac{\partial}{\partial #1} (#2)}

% Integral dx
\newcommand{\dx}{\mathrm{d}x}

% Alias for the Solution section header
\newcommand{\solution}{\textbf{\large Solution}}

% Probability commands: Expectation, Variance, Covariance, Bias
\newcommand{\E}{\mathrm{E}}
\newcommand{\Var}{\mathrm{Var}}
\newcommand{\Cov}{\mathrm{Cov}}
\newcommand{\Bias}{\mathrm{Bias}}

% 我的newcommand
\newcommand{\degree}{^{\circ}}
\newcommand{\arrow}{-{Stealth[length=4mm,width=2mm]}}
\newcommand{\rmd}{\mathrm{d}}
\newcommand{\deriv}[2]{\frac{\rmd #1}{\rmd #2}}
\renewcommand{\parallel}{\mathrel{/\mskip-2.5mu/}}
\newcommand{\parallelogram}{
	\mathord
    {\text
        {
			\tikz[baseline]
			\draw (0,.1ex) -- (.8em,.1ex) -- (1em,1.6ex) -- (.2em,1.6ex) -- cycle;
        }
    }
}

\begin{document}

\maketitle

\pagebreak

% 设置页码格式是罗马数字
\pagenumbering{roman}

% 生成目录
\tableofcontents

\pagebreak

% 设置页码格式是阿拉伯数字
\pagenumbering{arabic}

\pagebreak

\section{多元函数的基本概念}

\subsection{平面点集}

\subsubsection{坐标平面}

    建立了坐标系的平面。二元有序实数组\(\left(x,y\right)\)
    的全体,即$\mathbf{R}^2=\mathbf{R} \times \mathbf{R}=\{(x, y) 
    \mid x, y \in \mathbf{R}\}$就表示坐标平面。

\subsubsection{平面点集}

    坐标平面上具有某种性质\(P\)的点的几何,称作平面点集,记作

    \[
        E = \left\{\left(x,y\right) \mid \left(x,y\right)
        \text{具有某种性质}P\right\}
    \]

\subsubsection{邻域}

    设\(P_{0}\left(x_0,y_0\right)\)是\(xOy\)平面上一点,
    \(\delta\)是某一正数,与点\(P_{0}\left(x_0,y_0\right)\)
    距离小于\(\delta\)的点\(P\left(x,y\right)\)的全体,称为
    \(P_{0}\)的\(\delta\)邻域,记作\(U\left(P_0,\delta\right)\),
    即

    \[
        U\left(P_0,\delta\right) = \left\{\left(x,y\right)
        \mid \left|PP_0\right| < \delta\right\}
    \]

    或

    \[
        U\left(P_0,\delta\right) = \left\{\left(x,y\right)
        \mid \sqrt{\left(x-x_0\right)^2 + \left(y-y_0\right)^2} < \delta\right\}
    \]

\subsubsection{注意}

    \begin{enumerate}
        \item 点\(P_0\)的去心邻域,记作$\stackrel{\circ}{U}\left(P_0, \delta\right)$,即
            $$
                \stackrel{\circ}{U}\left(P_0, \delta\right)=\left\{P\mid0<\left|P P_0\right|<\delta\right\}
            $$
        \item 若不强调\(\delta\),也可记作\(U\left(P_0\right)\),
            \(\stackrel{\circ}{U}\left(P_0\right)\)
    \end{enumerate}

\subsubsection{利用点与点集的关系}

    若有一点\(P \in \mathbf{R}^2\),任意点集\(E \subset \mathbf{R}^2\)

    \begin{enumerate}
        \item 内点:\(\exists U\left(P\right)\),使\(U\left(P\right) \subset E\),
            则\(P\)为\(E\)的内点。
        \item 外点:\(\exists U\left(P\right)\),使\(U\left(P\right) \cap E = \phi\),
            则\(P\)为\(E\)的内点。
        \item 边界点:\(\forall U\left(P\right)\),若\(U\left(P\right)\)
            即有属于\(E\)的点,又有不属于\(E\)的点,则\(P\)为\(E\)的边界点。
        \item \(E\)的边界:\(E\)的边界点的全体,记作\(\partial E\)
    \end{enumerate}

\subsubsection{聚点}

    如果对于任意给定的\(\delta>0\),点\(P\)的去心邻域
    \(\stackrel{\circ}{U}\left(P,\delta\right)\)内总有\(E\)
    中的点,那么称\(P\)是\(E\)的聚点。

    例如,若\(E = \left\{\left(x,y\right) \mid
    1 < x^2+y^2 \leq 2\right\}\)。则\(x^2+y^2=1\)
    和\(x^2+y^2=2\)都是\(E\)的聚点。

\subsubsection{由点集所属类的特征分类}

    \begin{enumerate}
        \item 开集:若点集\(E\)中的所有点都是\(E\)的内点,则称\(E\)为开集;
        \item 闭集:若点集\(E\)的边界\(\partial E \in E\),则称\(E\)为闭集。
    \end{enumerate}

    例如,\(\left\{\left(x,y\right) \mid 1 < x^2+y^2 < 2\right\}\)为开集,
    \(\left\{\left(x,y\right) \mid 1 \leq x^2+y^2 \leq 2\right\}\)为闭集,\\
    \(\left\{\left(x,y\right) \mid 1 < x^2+y^2 \leq 2\right\}\)

\end{document}
