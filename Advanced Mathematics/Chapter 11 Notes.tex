%!TEX program = xelatex
\documentclass[12pt, a4paper]{article}

\usepackage[dvipsnames]{xcolor}

\usepackage{fancyhdr}
\usepackage{extramarks}
\usepackage{amsmath}
\usepackage{empheq}
\usepackage{amsthm}
\usepackage{amsfonts}
\usepackage{tikz}
\usepackage{tikz-3dplot}
\usepackage[plain]{algorithm}
\usepackage{algpseudocode}

\usepackage{ctex}
\usepackage{upgreek}
\usepackage{indentfirst}
\usepackage{wrapfig}
\usepackage{subfigure}
\usepackage{pgfplots}
\usepgfplotslibrary{patchplots}
\usepgfplotslibrary{colormaps}
\usepgfplotslibrary{colorbrewer}
\pgfplotsset{compat=1.18}
\usetikzlibrary{automata,positioning,shapes.geometric,arrows.meta,patterns,calc}
\numberwithin{equation}{section}
\CTEXoptions[today=old]

%
% Basic Document Settings
%

\topmargin=-0.25in
\evensidemargin=0in
\oddsidemargin=0in
\textwidth=6.5in
\textheight=9.2in
\headsep=0.25in

\linespread{1.1}

\pagestyle{fancy}
\lhead{\hmwkAuthorName}
\chead{\hmwkClass : \hmwkTitle}
\rhead{\firstxmark}
\lfoot{\lastxmark}
\cfoot{\thepage}

\renewcommand\headrulewidth{0.4pt}
\renewcommand\footrulewidth{0.4pt}

\setlength{\parindent}{2em}  % 2em代表首行缩进两个字符

%
% Create Problem Sections
%

\newcommand{\enterProblemHeader}[1]{
    \nobreak\extramarks{}{Problem \arabic{#1} continued on next page\ldots}\nobreak{}
    \nobreak\extramarks{Problem \arabic{#1} (continued)}{Problem \arabic{#1} continued on next page\ldots}\nobreak{}
}

\newcommand{\exitProblemHeader}[1]{
    \nobreak\extramarks{Problem \arabic{#1} (continued)}{Problem \arabic{#1} continued on next page\ldots}\nobreak{}
    \stepcounter{#1}
    \nobreak\extramarks{Problem \arabic{#1}}{}\nobreak{}
}

% \setcounter{secnumdepth}{0}
\newcounter{partCounter}
\newcounter{homeworkProblemCounter}
\setcounter{homeworkProblemCounter}{0}
% \nobreak\extramarks{Problem \arabic{homeworkProblemCounter}}{}\nobreak{}

%
% Homework Problem Environment
%
% This environment takes an optional argument. When given, it will adjust the
% problem counter. This is useful for when the problems given for your
% assignment aren't sequential. See the last 3 problems of this template for an
% example.
%
\newenvironment{homeworkProblem}[1][-1]{
    \ifnum#1>0
        \setcounter{homeworkProblemCounter}{#1}
    \fi
    \section{Problem \arabic{homeworkProblemCounter}}
    \setcounter{partCounter}{1}
    \enterProblemHeader{homeworkProblemCounter}
}{
    \exitProblemHeader{homeworkProblemCounter}
}

%
% Homework Details
%   - Title
%   - Due date
%   - Class
%   - Section/Time
%   - Instructor
%   - Author
%

\newcommand{\hmwkTitle}{Line Integrals and Surface Integrals}
\newcommand{\hmwkClass}{Advanced Mathematics}
\newcommand{\hmwkClassTime}{}
\newcommand{\myUniversiy}{Wuhan University}
\newcommand{\hmwkAuthorName}{\textbf{Lai Wei}}

%
% Title Page
%

\title{
    \vspace{2in}
    \textmd{\textbf{\hmwkClass:\ \hmwkTitle}}\\
    \vspace{0.4in}
    \large{\textit{\myUniversiy}}
    \vspace{3in}
}

\author{\hmwkAuthorName}
\date{\today}

\renewcommand{\part}[1]{\textbf{\large Part \Alph{partCounter}}\stepcounter{partCounter}\\}

%
% Various Helper Commands
%

% Useful for algorithms
\newcommand{\alg}[1]{\textsc{\bfseries \footnotesize #1}}

% % For derivatives
% \newcommand{\deriv}[1]{\frac{\mathrm{d}}{\mathrm{d}x} (#1)}

% For partial derivatives
% \newcommand{\pderiv}[2]{\frac{\partial}{\partial #1} (#2)}

% Integral dx
\newcommand{\dx}{\mathrm{d}x}

% Alias for the Solution section header
\newcommand{\solution}{\textbf{\large Solution}}

% Probability commands: Expectation, Variance, Covariance, Bias
\newcommand{\E}{\mathrm{E}}
\newcommand{\Var}{\mathrm{Var}}
\newcommand{\Cov}{\mathrm{Cov}}
\newcommand{\Bias}{\mathrm{Bias}}

% 我的newcommand
\newcommand{\degree}{^{\circ}}
\newcommand{\arrow}{-{Stealth[length=4mm,width=2mm]}}
\newcommand{\rmd}{\mathrm{d}}
\newcommand{\deriv}[2]{\frac{\rmd #1}{\rmd #2}}
\renewcommand{\parallel}{\mathrel{/\mskip-2.5mu/}}
\newcommand{\pderiv}[2]{\frac{\partial #1}{\partial #2}}
\newcommand{\parallelogram}{
	\mathord
    {\text
        {
			\tikz[baseline]
			\draw (0,.1ex) -- (.8em,.1ex) -- (1em,1.6ex) -- (.2em,1.6ex) -- cycle;
        }
    }
}

\begin{document}

\maketitle

\pagebreak

% 设置页码格式是罗马数字
\pagenumbering{roman}

% 生成目录
\tableofcontents

\pagebreak

% 设置页码格式是阿拉伯数字
\pagenumbering{arabic}

\pagebreak

\section{对弧长的曲线积分}

\subsection{对弧长的曲线积分的概念与性质}

\subsubsection{定义}

    设L为\(xOy\)面内的一条光滑曲线弧,函数\(f\left(x,y\right)\)在\(L\)上有界。
    在\(L\)上任意插入一点列\\\(M_1, M_2, \cdots M_n-1\)把\(L\)分成\(n\)个小段。
    设第\(i\)个小段的长度为\(\Delta s_i\);又\(\left(\xi_i, \eta_i\right)\)为第\(i\)个小段上任意取定的一点,
    作乘积\(f\left(\xi_i, \eta_i\right) \Delta s_i\)(\(i=1,2,\cdots,n\))并作和\(\sum_{i=1}^{n}f\left(\xi_i, \eta_i\right) \Delta s_i\)。
    如果当各小弧段的长度的最大值\(\lambda \rightarrow 0\)时,这和的极限总存在,
    且与曲线弧\(L\)的分法及点\(\left(\xi_i, \eta_i\right) \)的取法无关,
    那么称此极限为函数\(f\left(x,y\right)\)在曲线弧\(L\)上对弧长的曲线积分或第一类曲线积分,
    记作\({\displaystyle \int_{L} f\left(x,y\right) \rmd s}\),即
    \[\int_L f(x, y) \mathrm{d} s=\lim _{\lambda \rightarrow 0} \sum_{i=1}^n f\left(\xi_i, \eta_i\right) \Delta s_i\]
    其中\(f\left(x,y\right)\)叫做被积函数,\(L\)叫做积分弧段。

    如果\(L\)(或\(\varGamma\))是分段光滑的,我们规定函数在\(L\)(或\(\varGamma\))上的曲线积分等于函数在光滑的各段上的曲线积分之和。
    例如,设\(L\)可分成两段光滑曲线弧\(L_1\)及\(L_2\)(记作\(L=L_1+L_2\)),就规定
    \[\int_L f(x, y) \mathrm{d} s=\lim _{\lambda \rightarrow 0} \sum_{i=1}^n f\left(\xi_i, \eta_i\right) \Delta s_i\]


\subsubsection{性质}

    \begin{enumerate}
        \item 设\(\alpha\)、\(\beta\)为常数,则\[\int_L\left[\alpha f(x, y)+\beta g(x, y)\right] \mathrm{d} s=\alpha \int_L f(x, y) \mathrm{d} s+\beta \int_L g(x, y) \mathrm{d} s\]
        \item 若积分弧段$L$可分成两段光滑曲线弧$L_1$和$L_2$,则\[\int_L f(x, y) \mathrm{d} s=\int_{L_1} f(x, y) \mathrm{d} s+\int_{L_2} f(x, y) \mathrm{d} s\]
        \item 设在\(L\)上\(f\left(x,y\right) \leq g\left(x,y\right)\),则\[\int_L f(x, y) \mathrm{d} s \leq \int_L g(x, y) \mathrm{d} s\]特别地,
            有\[\left| \int_L f(x, y) \mathrm{d} s \right| \leq \int_L \left| f(x, y) \right| \mathrm{d} s\]
    \end{enumerate}


\subsection{对弧长的曲线积分的计算法}

    设$f(x, y)$在曲线弧$L$上有定义且连续,$L$的参数方程为
    \[
        \left\{\begin{array}{l}
        x=\varphi(t), \\
        y=\psi(t)
        \end{array} \quad(\alpha \leq t \leq \beta),\right.
    \]
    若$\varphi(t)$、$\psi(t)$在$[\alpha, \beta]$上具有一阶连续导数,且$\varphi^{\prime 2}(t)+\psi^{\prime 2}(t) \neq 0$,则曲线积分$\int_i f(x, y) \mathrm{d} s$存在,
    且\begin{equation}
    \int_L f(x, y) \mathrm{d} s=\int_a^\beta f[\varphi(t), \psi(t)] \sqrt{\varphi^{\prime 2}(t)+\psi^{\prime 2}(t)} \mathrm{d} t \quad(\alpha<\beta)
    \label{11-1-1}
    \end{equation}

    公式\ref{11-1-1}表明,计算对弧长的曲线积分\(\int_L f(x, y) \mathrm{d} s\)时,只要把\(x\)、\(y\)、\(\rmd s\)依次换为
    \(\varphi\left(t\right)\)、\(\psi\left(t\right)\)、\(\sqrt{\varphi^{\prime 2}(t)+\psi^{\prime 2}(t)}\),然后从\(\alpha\)
    到\(\beta\)作定积分就行了,这里必须注意,定积分的 下限\(\alpha\)一定要小于上限\(\beta\)。

    如果曲线弧长\(L\)由方程\[y=\psi(x) \quad (x_0 \leq x \leq X)\]给出,那么可以把这种情形看作是特殊的参数方程
    \[x=t, y=\psi(t) \quad (x_0 \leq t \leq X)\]的情形,从而由公式\ref{11-1-1}得出

    \begin{equation}
        \int_L f(x, y) \mathrm{d} s=\int_{x_0}^X f[x, \psi(x)] \sqrt{1+\psi^{\prime 2}(x)} \mathrm{d} x\left(x_0<X\right)
    \end{equation}

    类似地,如果曲线弧长\(L\)由方程\[x=\varphi(y) \quad (y_0 \leq y \leq Y)\]给出,那么有

    \begin{equation}
        \int_L f(x, y) \mathrm{d} s=\int_{y_0}^Y f[\varphi(y), y] \sqrt{1+\varphi^{\prime 2}(y)} \mathrm{d} y\left(y_0<Y\right)
    \end{equation}

    公式\ref{11-1-1}可推广到空间曲线弧\(\varGamma\)由参数方程\[x=\varphi(t), y=\psi(t), z=\omega(t) \quad(\alpha \leq t \leq \beta)\]给出的情形,
    这时有

    \begin{equation}
        \int_\varGamma f(x, y, z) \mathrm{d} s=\int_a^\beta f[\varphi(t), \psi(t), \omega(t)] \sqrt{\varphi^{\prime 2}(t)+\psi^{\prime 2}(t)+\omega^{\prime 2}(t)} \mathrm{d} t
        \quad (\alpha \leq \beta)
    \end{equation}

\end{document}
